%%%%%%%%%%%%%%%%%%%%%%%%%%%%%%%%%%%%%%%%%%%%%%%%%%%%%%%%%%%%%%%%%%%%%%%%%%%%%%%
%                          LaTeX beamer template                              %
%              © 2010 Valentin Roussellet (louen_AD_palouf.org)               %
%    Distributed under the terms of WTFPL v2, see http://sam.zoy.org/wtfpl    %
%%%%%%%%%%%%%%%%%%%%%%%%%%%%%%%%%%%%%%%%%%%%%%%%%%%%%%%%%%%%%%%%%%%%%%%%%%%%%%%
\documentclass{beamer}
% A reminder for beamer themes and colors
%                       Themes
% -----------------------------------------------------------------------------
% * Warsaw : black / colored toc above. Most commoly used
% * Copenhagen : like Warsaw without shadows
% * Marburg : black to blue right-sided toc
% * Berlin : colored panels above the text
% * Antibes : a lot like berlin
% * PaloAlto : colored left-sided toc
%                       Colors
%------------------------------------------------------------------------------
%   Name               background       titles          text
% * orchid (default)   white            dark blue       black
% * albatross          blue             light blue      white
% * beetle             grey             white           black
% * crane              white            black/yellow    black
% ============================== PREAMBLE =====================================
% Beamer settings ----------------
\mode<presentation>{
  %\usetheme{Marburg}                            % theme (see above)
  \usetheme{PaloAlto}                            % theme (see above)
  \usecolortheme{crane}                        % theme color (id)
  \setbeamercovered{transparent}                % For transparent zones
}

% Encoding and fonts -------------
\usepackage[utf8]{inputenc}                     % Input encoding
\usepackage[T1]{fontenc}                        % Output encoding
\usepackage{lmodern}                            % Modern font

% Language -----------------------
%\usepackage[right]{eurosym}                     % Symbol for euro

% Useful packages ----------------
\usepackage{listings}                           % for source code inclusion
\usepackage{hyperref}
% Mathematical symbols------------
\usepackage{amsmath}
\usepackage{amsfonts}
\usepackage{amssymb}

% Custom colors -----------------
\definecolor{codebg}{rgb}{0.97,0.97,0.97}
\definecolor{codecomment}{rgb}{0,0.5,0}
\definecolor{codestring}{rgb}{0.5,0,0}

% Listings setup -----------------
\lstset{language=C++,                             % programming language
%morekeywords={},                                % additionnal keywords
basicstyle=\small\ttfamily,                     % style of the code
keywordstyle=\color{blue}\bfseries,             % style of the keywords
stringstyle=\color{codestring},                 % style of the strings
commentstyle=\color{codecomment},               % style of the comments
showspaces=false,                               % do not underline spaces in code
showstringspaces=false,                         % do not underline spaces in strings
showtabs=false,                                 % do not underline tabs
numbers=left,                                   % where are the line-numbers
numberstyle=\tiny,                               % style of the line-numbers
backgroundcolor=\color{codebg},                 % background-color
stepnumber=1,                                   % the step between two line-numbers.
extendedchars=true,                             % allows extended characters
columns=flexible,                               % sets the columns to non-fixed width
tabsize=2,                                      % sets the tabulation width
frame=trBL,                                     % adds a frame around the code with double line on the bottom
frameround=tttt,                                % rounds the frame
breaklines=true,                                % line breaks automatically
breakautoindent=true,                           % keep indentation level withh line breaks
captionpos=b                                    % the caption is at the bottom
}

% Additional beamer commands ------
% this command prints a frame toc at each section beginning
\AtBeginSection[] {
  \begin{frame}
      \frametitle{What next ?}
      \tiny{\tableofcontents[currentsection]}
  \end{frame}
}

% ============================== FRONT MATTER =================================
\title{Magic : l'Assembleur}
\subtitle{Formation C++ bas niveau}
\author{Valentin \textsc{Roussellet}}
\institute[VIA]{Centrale Reseaux}
\date{\today}
\subject{}                                      % Information PDF
%\logo{\includegraphics[height=1cm]{logo.png}}  % Logo

% ============================= END OF PREAMBLE ===============================
\begin{document}
\begin{frame}
  \titlepage
\begin{center}
\href{mailto:louen@via.ecp.fr}{louen@via.ecp.fr}
\end{center}
\end{frame}
\section*{Introduction}
\begin{frame}
    \frametitle{Abandonnez tout espoir}
% image du desassembleur
\end{frame}

\begin{frame}
    \frametitle{Pourquoi connaitre l'assembleur}
Si vous vivez en 1895 (Nom de Zeus, Marty !)
\begin{itemize}
 \item Parce qu'il n'y a pas d'autre langage disponible
 \item Pour écrire du code super optimisé 
\end{itemize}
\pause
Mais en 2012 :
\begin{itemize}
 \item Pour comprendre ce qu'il se passe "sous le capot".
 \item Pour débugger quand on a pas la source (code optimisé par exemple)
\end{itemize}
\end{frame}

\begin{frame}
    \frametitle{Par ou commencer}
\begin{itemize}
 \item On va compiler du code C++ et regarder l'assembleur généré 
 \item J'utiliserai Visual Studio 2010 (syntaxe "Intel")
        \begin{itemize}
         \item Générer l'assembleur : \textbf{cl.exe /FA}
         \item Observer le code désassemblé : \textit{Debug > Windows > Disassembly}  
        \end{itemize}
        \pause
 \item Mais si vous êtes linuxien :
        \begin{itemize}
         \item Générer l'assembleur : \textbf{gcc -S}
         \item Observer le code désassemblé : \textbf{disass} dans gdb
         \item \textbf{-masm=intel} pour GCC ou \textbf{set disassembly-flavor intel} dans GDB pour avoir la syntaxe Intel  
        \end{itemize}
\pause
\item A la fin, vous en saurez un peu plus sur les \textbf{types}, les \textbf{pointeurs}, les \textbf{fonctions} et la \textbf{pile}, comment sont gérés les \textbf{objets} C++, et beaucoup d'autres choses.
\end{itemize}
\end{frame}
\begin{frame}
    \frametitle{Sommaire}
    \tiny{\tableofcontents}
\end{frame}

\section{Avengers, Disassemble !}
\begin{frame}[fragile]
  \frametitle{Notre premier désassemblage}

    \begin{lstlisting}
int x = 1;
int y = 2;
int z = 0;

z = x + y;
    \end{lstlisting}
\pause
    \begin{lstlisting}[language={[x86masm]Assembler}]
     ;3:         int x = 1;
00CB139E  mov         dword ptr [ebp-8],1  
     ;4:         int y = 2;
00CB13A5  mov         dword ptr [ebp-14h],2  
     ;5:         int z = 0;
00CB13AC  mov         dword ptr [ebp-20h],0  
     ;6: 
     ;7:         z = x + y;
00CB13B3  mov         eax,dword ptr [ebp-8]  
00CB13B6  add         eax,dword ptr [ebp-14h]  
00CB13B9  mov         dword ptr [ebp-20h],eax  
    \end{lstlisting}

\end{frame}
\begin{frame} [fragile]
\frametitle{Qu'est-ce que ca veut dire ?}
\begin{description}
\item[add] Instruction d'addition (you don't say ?)
\item[mov] Instruction qui copie de la mémoire : \lstinline[language={[x86masm]Assembler}]+mov dest src+
\item[eax] C'est un \textbf{registre} (mémoire embarquée du processeur). Les choses intéressantes se passent dans des registres.
\item[ebp] Un autre registre, appelé \textbf{base pointer}. On l'utilise pour accéder aux variables locales en mémoire.
\item[dword ptr] Valeur du mot mémoire de 32 bits situé a l'adresse entre crochets.
\end{description}
\end{frame}

\begin{frame} [fragile]
\frametitle{Qu'est-ce que ca veut dire ?}
    \begin{lstlisting}[language={[x86masm]Assembler}]
 ; la memoire a l'adresse ebp-8 (adresse de x) prend la valeur 1
mov         dword ptr [ebp-8],1
 ; la memoire a ebp-14h (hexa) prend la valeur 2 (c'est y).
mov         dword ptr [ebp-14h],2
 ; idem pour z. 
mov         dword ptr [ebp-20h],0
 ; charge le contendu de [ebp-8] (x) dans eax 
mov         eax,dword ptr [ebp-8]
 ; additione eax avec le contenu de [ebp-14h] (y)
add         eax,dword ptr [ebp-14h]
 ; deplace la valeur de eax dans [ebp-20h] (z)
mov         dword ptr [ebp-20h],eax
    \end{lstlisting}
\end{frame}

\section{Casse toi, pauvre type !}

\begin{frame}
\frametitle{Petit rappel sur les types} 

\begin{itemize}

\item En C : types \textit{fondamentaux} (mots-clefs du langage) 
\item 3 catégories : \textbf{void}, \textbf{entiers} et \textbf{nombres en virgule flottante}.
\item Chaque type a des \textit{specifications} déterminant leur représentation dans les types \textit{intrinseques} du CPU.
\pause 
\item Les types \textit{dérivés} : \textbf{pointeur} sur un type, \textbf{structure} ou \textbf{union} de plusieurs types.


\end{itemize}

\end{frame}

\begin{frame}
\frametitle{Les types entiers}
\begin{itemize}
\item Par ordre croissant de taille \lstinline+ char, short, int, long, long long+
\item Plus les variantes \lstinline+unsigned,  signed+ (quel est le défaut~?).
\item \lstinline+sizeof(char) == 1+ (et au moins 8 bits).
\item \lstinline+char*+ permet d'accéder a toute la mémoire.
\pause
\item Sur x86/Visual Studio 
\begin{description}
  \item[char] 1 octet ($-128$ a $127$ ou $0$ a $255$)
  \item[short] 2 octets ( $-32~768$ a $32~767$ ou $0$ a $65535$)
  \item[int] 4 octets ($-2~147~483~ 648$ a $2~147~483~ 647$ ou $0$ a $4~294~967~296$)
  \item[long] Identique a int.
  \item[long long] 8 octets (je vous laisse faire le calcul).
\end{description}

\end{itemize}
\end{frame}

\begin{frame}[fragile]
\frametitle{Types intrinseques}

\begin{itemize}
\item Les types effectivement manipulés par le CPU : \textbf{intrinseques}
\item Sous x86 : \lstinline[language={[x86masm]Assembler}]+dword+ (32 bits) (\lstinline+word+ étant 16 bits)
\item Un type de 64 bits (\lstinline+long long+) sera représenté par deux \lstinline[language={[x86masm]Assembler}]+dword+s.
\end{itemize}

\begin{lstlisting}[language={[x86masm]Assembler}]
    ;10:   long long l = 1;
00E0139E  mov         dword ptr [ebp-0Ch],1  
00E013A5  mov         dword ptr [ebp-8],0  
    ;11:         long long m = 2;
00E013AC  mov         dword ptr [ebp-1Ch],2  
00E013B3  mov         dword ptr [ebp-18h],0  
    ;12:         long long n = l+m;
00E013BA  mov         eax,dword ptr [ebp-0Ch]  
00E013BD  add         eax,dword ptr [ebp-1Ch]  
00E013C0  mov         ecx,dword ptr [ebp-8]  
00E013C3  adc         ecx,dword ptr [ebp-18h]  
00E013C6  mov         dword ptr [ebp-2Ch],eax  
00E013C9  mov         dword ptr [ebp-28h],ecx 
\end{lstlisting}

\end{frame}

\begin{frame}
\frametitle{Représentation des entiers}
\begin{itemize}
\item \lstinline+unsigned+ : valeur binaire\\
exemple : \lstinline+unsigned short x = 0x8A2F+ = $35~375$

\item \lstinline+signed+ : \textbf{complément a deux} $ -i  = NOT(i) + 1 = 2^n - i $
 \begin{tabular}{c c c}
 127 & = & 0x7F\\
 ...& & \\
 2 & = & 0x02\\
 1 & = & 0x01\\
 0 & = & 0x00\\
-1 & = & 0xFF\\
-2 & = & 0xFE \\
...& &  \\
-128 & =&  0x80
\end{tabular}

\item Le type \lstinline +bool+ : pas de taille précisée, seulement 2 valeurs. (C++ seulement). 

\end{itemize}
\end{frame}

\begin{frame}
\frametitle{Représentation des réels}
\begin{itemize}
 \item Norme IEEE 754.
 \item \lstinline+float+ (4 octets), \lstinline+double+ (8 octets) \lstinline+long double+ (8 ou 10 octets)
\end{itemize}
\begin{tiny}
\begin{tabular} {|@{}c@{}|@{}c@{}|@{}c@{}|@{}c@{}|@{}c@{}|@{}c@{}|@{}c@{}|@{}c@{}|@{}c@{}|@{}c@{}|@{}c@{}|@{}c@{}|@{}c@{}|@{}c@{}|@{}c@{}|@{}c@{}|@{}c@{}|@{}c@{}|@{}c@{}|@{}c@{}|@{}c@{}|@{}c@{}|@{}c@{}|@{}c@{}|@{}c@{}|@{}c@{}|@{}c@{}|@{}c@{}|@{}c@{}|@{}c@{}|@{}c@{}|@{}c@{}|}
\hline

00&01&02&03&04&05&06&07&08&09&10&11&12&13&14&15&16&17&18&19&20&21&22&23&24&25&26&27&28&29&20&31\\
\hline
s & 
\multicolumn{8}{|c|}{exposant} & 
\multicolumn{23}{|c|}{mantisse} \\
\hline
\end{tabular}
\end{tiny}
\begin{itemize}
\item $val = s \times 2^{\mathrm{exposant}}\times \mathrm{mantisse}$
\item mantisse : chaque bit = $\frac{1}{2^b}$
\item Il y a donc $+0$ et $-0$
\item Ainsi que $+\infty$ et  $-\infty$
\item Des valeurs NaN et dénormalisées.
\item Attention a la précision (et a \lstinline+printf+)
\end{itemize}
\end{frame}


\begin{frame}
\frametitle{Les pointeurs}
\begin{itemize}
\item Les pointeurs \textbf{de données} ont tous la meme taille.
\item La taille des pointeurs définit l'\textit{espace adressable} 
\item Sur win32 : 32 bits (4Gib) -- mais ce n'est pas toujours la meme taille que \lstinline+int+ !
\end{itemize}

\end{frame}

\begin{frame}
\frametitle{Représentation mémoire : l'endianness}
\begin{itemize}
\item Ordre des octets : \textbf{big-endian} et \textbf{little-endian}  
\item Sur PC x86 (et x64) : c'est du little-endian.
\item D'autres machines sont big-endian : POWERPC (vieux Mac, XBox 360), Sun SPARC... 
\pause
\item On va regarder la mémoire pour voir comment est représenté un \lstinline+int+ 
\item Sous VS : \textit {Debug>Windows>Memory} (gdb : \textit{display})
\end{itemize}
\pause
\begin{itemize}
\item code : \lstinline+unsigned int x = 0x12345678;+
\item asm : \lstinline[language={[x86masm]Assembler}]+mov dword ptr [ebp-8],12345678h+
\item mémoire @ebp - 8 : \texttt{0x004BF848  78 56 34 12}
\end{itemize}

% parallele avec les dates.


\end{frame}
\begin{frame}[fragile]
\frametitle{Alignement et remplissage}
\begin{itemize}
\item Les données sont alignées sur la taille d'un mot.
\end{itemize}
\begin{columns}
  \begin{column}{0.3\textwidth}
\begin{lstlisting}

struct Test {
  int a;
  char b;
  long c;
};
\end{lstlisting}
  \end{column}
  \begin{column}{0.5\textwidth}
\begin{lstlisting}
int main() {
  struct Test inst;
  inst.a = 42;
  inst.b = '@';
  inst.c = 0x12345678;
}
\end{lstlisting}
  \end{column}
\end{columns}
\pause
\begin{itemize}
\item Résultat dans la mémoire : 
\end{itemize}
\begin{lstlisting}
0x003EF8FC  2a 00 00 00  // a (32 b)  
0x003EF900  40 cc cc cc  // b (8 b) + 24 b de 'padding' 
0x003EF904  78 56 34 12  // c (32b) 
\end{lstlisting}

\end{frame}

\begin{frame}[fragile]
\frametitle{Alignement et remplissage}
\begin{itemize}
\item On peut changer le comportement du compilateur 
\end{itemize}
\begin{columns}
  \begin{column}{0.3\textwidth}
\begin{lstlisting}
#pragma pack(1)
struct Test {
  int a;
  char b;
  long c;
};
\end{lstlisting}
  \end{column}
  \begin{column}{0.5\textwidth}
\begin{lstlisting}
int main() {
  struct Test inst;
  inst.a = 42;
  inst.b = '@';
  inst.c = 0x12345678;
}
\end{lstlisting}
  \end{column}
\end{columns}
\pause
\begin{itemize}
\item Résultat dans la mémoire : 
\end{itemize}
\begin{lstlisting}
0x003EF8FC  2a 00 00 00  // a (32 b)  
0x003EF904  40 78 56 34  // b (8b) + 24 premiers b de c 
0x003EF900  12 cc cc cc  // fin de c 
\end{lstlisting}
\end{frame}
\begin{frame}
\frametitle{Alignement et remplissage}
\begin{itemize}
\item Sous x86 on peut accéder aux données meme non-alignées
\item ...sauf dans certains cas (SIMD par exemples).
\item Le compilateur aligne par défaut et remplit les structures avec du vide
\item ... mais on peut le contraindre : attention a la portabilité !
\end{itemize}
\end{frame}


\section{Sexe a Pile}

\begin{frame}
\frametitle{A la recherche de La Pile}
Ce que vous devriez savoir de La Pile (\textit{stack}).
\begin{itemize}
\item C'est une pile (surprise !)
        \begin{itemize}
        \item Il y a donc les opérations \texttt{push()}, \texttt{pop()} et  \texttt{top()}
        \end{itemize}
\item Elle contient les variables \textbf{automatiques} 
\item C'est la que sont passés les \textbf{arguments} des fonctions (\textit{stack frame})
\end{itemize}
\end{frame}


\begin{frame}
\frametitle{Stack frame et appel de fonction}
Que se passe t'il quand on appelle une fonction ?
\begin{itemize}
\item L'état courant du processeur (i.e. les valeurs des registres) est sauvé sur la Pile.
\item Les arguments de la fonction appelée sont positionnés sur la Pile
\item L'ensemble de ces données : le \textbf{stack frame}.
\item Le CPU "saute" a l'adresse de la fonction appelée.
\end{itemize}
Si on break, on peut voir les stack frames des fonctions empilés :
\begin{itemize}
\item Valeur des parametres et variables locales
\item Contenu des registres utilisés au moment ou l'appel suivant a été fait
\item Adresse de retour de la fonction apres \lstinline+return+
\end{itemize} 
\end{frame}

\begin{frame}
\frametitle{Les conventions d'appel} % mettr ça plus haut ?
La convention d'appel standard de VS/x86 s'appelle \texttt{stdcall}
\begin{itemize}
\item Ou vont les paramètres ? Sur la pile !
\item Ou est la valeur de retour ? dans \texttt{eax}
\item Qui vide la pile après l'appel ? l'appelé.
\item Qui sauve quels registres ? l'appelant (\texttt{eax}-\texttt{edx}) et l'appelé le reste
\end{itemize}
\end{frame}
% TODO  : image d'un stack frame ?

\begin{frame}[fragile]
\frametitle{Désassemblons !}
Un petit code tout innoffensif :
\begin{lstlisting}
int ajouterUn( int argument ){
  int local = argument + 1;
  return local;
}

int main() {
  int result = 0;
  result = ajouterUn(result);
  return 0;
}
\end{lstlisting}
\begin{itemize}
\item On met un point d'arret ligne 6 (\lstinline+int main()+)...
\end{itemize}
\end{frame}

\begin{frame}[fragile]
\frametitle{Désassemblons !}
\begin{lstlisting}[language={[x86masm]Assembler}, basicstyle={\scriptsize\ttfamily}]
push        ebp  
mov         ebp,esp  
sub         esp,44h  
push        ebx  
push        esi  
push        edi  
;     7:         int result = 0;
mov         dword ptr [ebp-4],0  
;     8:         result = ajouterUn(result);
mov         eax,dword ptr [ebp-4]  
push        eax  
call        00C81041  
add         esp,4  
mov         dword ptr [ebp-4],eax  
;     9:         return 0;
xor         eax,eax  
    10: }
pop         edi  
pop         esi  
pop         ebx  
mov         esp,ebp  
pop         ebp  
ret  
\end{lstlisting}
\end{frame}
\begin{frame}[fragile]
\frametitle{Don't panic !}
\begin{description}
\item[esp]  \textbf{Stack Pointer} : pointe en "haut" de la Pile 
\item[ebp]  \textbf{Base Pointer} : pointe au début du stack frame courant.
\end{description}
\begin{itemize}
\item Les variables sur la Pile sont référencées par rapport a \texttt{ebp} : \lstinline[language={[x86masm]Assembler}]+dword ptr [ebp-4]+ 
\item Mais alors pourquoi des soustractions ?
\pause
\item Parce que la pile grandit vers le bas ! (le "haut" a une adresse plus petite que le "bas").
\end{itemize}
\end{frame}

\begin{frame}[fragile]
\frametitle{Push it, pop it}
\begin{itemize}
\item Quand on \texttt{push} : \texttt{esp} est décrémenté, et l'opérande est stocké dans l'espace pointé par \texttt{esp}
\item \texttt{esp} pointe donc sur la derniere valeur ajoutée dans la stack
\item Quand on pop, l'a mémoire par \texttt{esp} est copiée dans un registre et \texttt{esp} est incrémenté (la pile diminue)
\end{itemize}
\pause
En résumé :
\begin{itemize}
\item  \lstinline[language={[x86masm]Assembler}]+push ecx+ $\rightarrow$ \lstinline+esp --; *esp = ecx;+ 
\item  \lstinline[language={[x86masm]Assembler}]+pop ebx+ $\rightarrow$  \lstinline!ebx = *esp; esp++;!
\end{itemize}
\end{frame}



\begin{frame}[fragile]
\frametitle{Le préambule} 
\begin{lstlisting}[language={[x86masm]Assembler}, basicstyle={\scriptsize\ttfamily}]
push        ebp ; l'ancien bp est sauve
mov         ebp,esp  ; nouveau bp
sub         esp,44h  ; ajoute de l'espace
push        ebx  ; sauve les registres
push        esi  
push        edi  
\end{lstlisting}
\end{frame}

% image avant -> après 
\begin{frame}[fragile]
\frametitle{L'épilogue}
Il fait l'inverse du préambule :
\begin{lstlisting}[language={[x86masm]Assembler}, basicstyle={\scriptsize\ttfamily}]
013112A1  pop         edi
013112A2  pop         esi
013112A3  pop         ebx
013112A4  mov         esp,ebp ; restore esp a son ancienne valeur
013112A6  pop         ebp ; l'ancien bp est juste la !
013112A7  ret
\end{lstlisting}
\end{frame}

\begin{frame}[fragile]
\frametitle{Et notre code ?}
\begin{lstlisting}[language={[x86masm]Assembler}, basicstyle={\scriptsize\ttfamily}]
;     7:         int result = 0;
mov         dword ptr [ebp-4],0  ; result est a ebp-4 dans notre SF 
;     8:         result = ajouterUn(result);
mov         eax,dword ptr [ebp-4] ; copie le parametre  (via eax)
push        eax  ; pour le mettre 	sur la pile 
call        00C81041  ; saute a l'adresse de la fonction
add         esp,4  ; pareil que pop(mais ignore la valeur)
mov         dword ptr [ebp-4],eax  ; le resultat (eax) va dans result
;     9:         return 0;
xor         eax,eax  
\end{lstlisting}
\begin{itemize}
\item Le parametre est copié sur la pile (via un registre)
\item On sait que le résultat est dans \texttt{eax}.
\end{itemize}
\end{frame}

\begin{frame}[fragile]
\frametitle{Notre fonction}
Il est temps (enfin) de désassembler notre fonction !
\begin{lstlisting}[language={[x86masm]Assembler}, basicstyle={\scriptsize\ttfamily}]
; 1: int ajouterUn( int argument ){
push        ebp      ;preambule : on connait !
mov         ebp,esp  
sub         esp,44h  
push        ebx  
push        esi  
push        edi  
;2: 	int local = argument + 1;
mov         eax,dword ptr [ebp+8]  ;c'est hors de notre SF
add         eax,1  
mov         dword ptr [ebp-4],eax  ; variable locale dans notre SF
;3: 	return local;
mov         eax,dword ptr [ebp-4]  ; resultat dans eax
;4: }
pop         edi  ; epilogue : on connait aussi
pop         esi  
pop         ebx  
mov         esp,ebp  
pop         ebp  
ret  
\end{lstlisting}

\end{frame}

\begin{frame}[fragile]
\frametitle{Résumé}
La pile sert a :
\begin{itemize}
\item Stocker les variables locales
\item Sauver les registres et passer les parametres
\end{itemize}
La \textbf{convention d'appel} détermine qui fait quoi sur la pile.

Le \textbf{stack frame}
\begin{itemize}
\item Un par appel de fonction
\item Compris entre \texttt{esp} et \texttt{ebp}
\item Les arguments : dans le SF de la fonction appelante
\end{itemize}

\end{frame}

\section{Condition zéro}
\begin{frame}[fragile]
\frametitle{If I had a hammer}
\begin{itemize}
\item Démarrons avec un code tres facile :
\end{itemize}
\begin{lstlisting}
int f (int arg) {
  int local = 0;
  if ( arg < 0 ) {
    local = 1;
  }
  return 0;
}
\end{lstlisting}
\end{frame}

\begin{frame}[fragile]
\frametitle{If I had a hammer}
\begin{itemize}
\item Voila ce que donne le désassemblage : 
\end{itemize}
\begin{lstlisting}[language={[x86masm]Assembler}]
;2:         int local = 0;
00B91299  mov         dword ptr [ebp-4],0  
;3:         if ( arg < 0 ) {
00B912A0  cmp         dword ptr [ebp+8],0  
00B912A4  jge         00B912AD  
;4:                 local = 1;
00B912A6  mov         dword ptr [ebp-4],1  
;5:         }
;6:         return 0;
00B912AD  xor         eax,eax  
\end{lstlisting}
\end{frame}

\begin{frame}[fragile]
\frametitle{Que remarquons-nous ?}
\begin{itemize}
\item La ligne du \lstinline+if+ devient deux instructions
\begin{description} 
\item[cmp] comparaison (aucun effet mais change les EFLAGS)
\item[jge] Jump if Greater or Equal  : saute a l'adresse indiquée selon la valeur des EFLAGS
\end{description} 
\pause
\item C'est l'inverse de ce qu'on a écrit en C !
\pause
\item \textbf{si} \textit{condition} \textbf{alors} entrer dans les accolades
\item \textbf{si} $\neg$\textit{condition} \textbf{alors} sauter 
\end{itemize}


\begin{lstlisting}
int local = 0;
if ( arg >= 0 ) goto Positive;
local = 1;
Positive:
return 0;
\end{lstlisting}

\end{frame}

\section{Les boucles}
\begin{frame}
\frametitle{Once upon a while}
\end{frame}

% la pile

% les conditions
% les boucles
% modele objets (vtable, héritage, etc.)

% assembleur optimisé - inlining
%




%  \begin{block}{A block}
%   Text in a block
%  \end{block}
%  \begin{exampleblock}{Another block}
%   Text in an example block
%  \end{exampleblock}
%  \pause
%  \begin{alertblock}{Yet another block}
%   Text in an warning block
%  \end{alertblock}

% WARNING : every frame with code in it must be declared [fragile]
% and the \end{frame} should not be indented (see beamer documentation)
\end{document}
