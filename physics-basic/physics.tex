%%%%%%%%%%%%%%%%%%%%%%%%%%%%%%%%%%%%%%%%%%%%%%%%%%%%%%%%%%%%%%%%%%%%%%%%%%%%%%%
%                          LaTeX beamer template                              %
%              © 2010 Valentin Roussellet (louen_AD_palouf.org)               %
%    Distributed under the terms of WTFPL v2, see http://sam.zoy.org/wtfpl    %
%%%%%%%%%%%%%%%%%%%%%%%%%%%%%%%%%%%%%%%%%%%%%%%%%%%%%%%%%%%%%%%%%%%%%%%%%%%%%%%
\documentclass{beamer}
% A reminder for beamer themes and colors
%                       Themes
% -----------------------------------------------------------------------------
% * Warsaw : black / colored toc above. Most commoly used
% * Copenhagen : like Warsaw without shadows
% * Marburg : black to blue right-sided toc
% * Berlin : colored panels above the text
% * Antibes : a lot like berlin
% * PaloAlto : colored left-sided toc
%                       Colors
%------------------------------------------------------------------------------
%   Name               background       titles          text
% * orchid (default)   white            dark blue       black
% * albatross          blue             light blue      white
% * beetle             grey             white           black
% * crane              white            black/yellow    black
% ============================== PREAMBLE =====================================
% Beamer settings ----------------
\mode<presentation>{
  %\usetheme{Marburg}                            % theme (see above)
  \usetheme{PaloAlto}                            % theme (see above)
  \usecolortheme{crane}                        % theme color (id)
  \setbeamercovered{transparent}                % For transparent zones
}

% Encoding and fonts -------------
\usepackage[utf8]{inputenc}                     % Input encoding
\usepackage[T1]{fontenc}                        % Output encoding
\usepackage{lmodern}                            % Modern font

% Language -----------------------
%\usepackage[right]{eurosym}                     % Symbol for euro
%\usepackage[french]{babel}

% Useful packages ----------------
\usepackage{listings}                           % for source code inclusion
\usepackage{hyperref}
\hypersetup{linkcolor=cyan}
\usepackage{movie15}

% Mathematical symbols------------
\usepackage{amsmath}
\usepackage{amsfonts}
\usepackage{amssymb}

% Custom colors -----------------
\definecolor{codebg}{rgb}{0.97,0.97,0.97}
\definecolor{codecomment}{rgb}{0,0.5,0}
\definecolor{codestring}{rgb}{0.5,0,0}

% Listings setup -----------------
\lstset{language=C++,                             % programming language
%morekeywords={},                                % additionnal keywords
basicstyle=\small\ttfamily,                     % style of the code
keywordstyle=\color{blue}\bfseries,             % style of the keywords
stringstyle=\color{codestring},                 % style of the strings
commentstyle=\color{codecomment},               % style of the comments
showspaces=false,                               % do not underline spaces in code
showstringspaces=false,                         % do not underline spaces in strings
showtabs=false,                                 % do not underline tabs
numbers=left,                                   % where are the line-numbers
numberstyle=\tiny,                               % style of the line-numbers
backgroundcolor=\color{codebg},                 % background-color
stepnumber=1,                                   % the step between two line-numbers.
extendedchars=true,                             % allows extended characters
columns=flexible,                               % sets the columns to non-fixed width
tabsize=2,                                      % sets the tabulation width
frame=trBL,                                     % adds a frame around the code with double line on the bottom
frameround=tttt,                                % rounds the frame
breaklines=true,                                % line breaks automatically
breakautoindent=true,                           % keep indentation level withh line breaks
captionpos=b                                    % the caption is at the bottom
}

% Additional beamer commands ------
% this command prints a frame toc at each section beginning
% \AtBeginSection[] {
%   \begin{frame}
%       \frametitle{What next ?}
%       \tiny{\tableofcontents[currentsection]}
%   \end{frame}
% }

% ============================== FRONT MATTER =================================
\title{Getting physical}
\subtitle{Introduction to physics engines}
\author{Charly \textsc{Mourglia} \& Valentin \textsc{Roussellet}}
\institute[IRIT]{IRIT}
\date{\today}
\subject{}                                      % Information PDF
%\logo{\includegraphics[height=1cm]{logo.png}}  % Logo

% ============================= END OF PREAMBLE ===============================
\begin{document}
\begin{frame}
  \titlepage
\begin{center}
\href{mailto:valentin.roussellet@irit.fr}{valentin.roussellet@irit.fr}
\end{center}
\end{frame}


\begin{frame}
 \frametitle{Introduction}
 \includegraphics[width=1.04\textwidth]{bullet.jpg}
\end{frame}

\begin{frame}
 \frametitle{Outline}
 \tableofcontents
 \begin{alertblock}{Warning}
  This presentation contains maths. Be afraid. Be very afraid.
 \end{alertblock}

\end{frame}



\section{Moving objects}
\subsection{Dynamics 101}
\begin{frame}
  \frametitle{Position, velocity, acceleration}
  \begin{itemize}
  \item Position $\mathbf{x}(t)$ of an object at time $t$
    $$
    \mathbf{x}(t) = \mathbf{x}_0 + \int_0^t \mathbf{v}(t)dt
    $$
    
    \pause
  \item Velocity $\mathbf{v}(t)$ of an object at time $t$
    $$
    \mathbf{v}(t) = \dot{\mathbf{x}}(t) = \frac{d}{dt}\mathbf{x}(t) = \mathbf{v}_0 + \int_0^t \mathbf{a}(t)dt
    $$

    \pause
  \item Acceleration $\mathbf{a}(t)$ of an object at time $t$
    $$
    \mathbf{a}(t) = \dot{\mathbf{v}}(t) = \ddot{\mathbf{x}}(t) = \frac{d^2}{dt^2}\mathbf{x}(t)
    $$
  \end{itemize}
\end{frame}

\begin{frame}
  \frametitle{Forces}
  Objects change their movement when force is applied to them.
  \begin{exampleblock}{Examples of forces}
  \begin{itemize}
  \item Friction
  \item Gravity
  \item Elastic collisions
  \end{itemize}
  \end{exampleblock}
\end{frame}
  
\begin{frame}
 \frametitle{Newton's second law}
  Relation between \textbf{forces} and \textbf{acceleration}~:

    \begin{block}{Newton's Second Law of Motion}
      ``The sum of the external forces acting on an object
      is equal to the mass of that object multiplied by
      the acceleration vector of the object''

      $$
      \mathbf{F_{ext}}(t) = \sum_{i=1}^N \mathbf{F}_i(t) \hspace{25pt} 
      \mathbf{F_{ext}}(t) = m \mathbf{a}(t)
      $$
      \pause
      $$
      \mathbf{a}(t) = \frac{1}{m} \mathbf{{F}_{ext}}(t)
      $$
    \end{block}
\end{frame}

\begin{frame}
 \frametitle{Newton's second law}
  Relation between \textbf{forces} and \textbf{acceleration}~:
  \begin{center}\includegraphics{puma.png}\end{center}
\end{frame}

\subsection{Numerical integration}
\begin{frame}
  \frametitle{Numerical integration}

  \begin{itemize}
  \item Finding the position of an object at a given time $t$ requires solving
    the following second-order differential equation
    
    $$
    \mathbf{F}_{ext}(t) = m\ddot{\mathbf{x}}(t)
    $$
    \pause
  \item Discretize the time and solve for the discretization steps $\delta t$
  \end{itemize}
  
\end{frame}

\begin{frame}
  \frametitle{Forward (explicit) Euler Method}

  \begin{itemize}
  \item Given timestep $\delta t$ we have

    $$
    \begin{cases}
      \mathbf{x}_{i+1} &= \mathbf{x}_i + \delta t \mathbf{v}_i\\
      \mathbf{v}_{i+1} &= \mathbf{v}_i + \delta t \mathbf{a}_i
    \end{cases}
    $$
    \pause
  \item Very fast to compute
    \pause
  \item Very inaccurate, can be unstable
  \item Not used in real-time physics engines due to the inaccuracies / instabilities
  \end{itemize}
\end{frame}



\begin{frame}
  \frametitle{Forward Euler Problems}

  \begin{itemize}
  \item Error~: approximately proportionnal to $\delta t$
    \pause
  \item Instability:~\\
    \begin{figure}[p]
      \centering
      \includegraphics[width=0.5\linewidth]{instability.jpg}
      \caption{Solutions for the DE $y' = -2.3y$ with $h=1$ (blue squares) and $h=0.7$ (red dots).
        The exact solution is $y(t) = e^{-2.3t}$}
    \end{figure}
  \end{itemize} 
\end{frame}

\begin{frame}
  \frametitle{Backward (implicit) Euler Method}

  \begin{itemize}
  \item Given timestep $\delta t$ we have

    $$
    \begin{cases}
      \mathbf{v}_{i+1} &= \mathbf{v}_i + \delta t \mathbf{a}_{i+1}\\
      \mathbf{x}_{i+1} &= \mathbf{x}_i + \delta t \mathbf{v}_{i+1}
    \end{cases}
    $$

  \item Stable solution with no limit on the timestep
  \item Absorbs energy (the larger the timestep, the more energy absorbed)
  \pause
  \item Needs to \textbf{solve} the next value for $\mathbf{a}$ (\emph{e.g Newton-Raphson}), very expensive
  \item Not used in real-time physics engines due to the cost. 
  \end{itemize}
\end{frame}

\begin{frame}
  \frametitle{Semi-implicit Euler Method}

  \begin{itemize}
  \item Given timestep $\delta t$ we have

    $$
    \begin{cases}
      \mathbf{v}_{i+1} &= \mathbf{v}_i + \delta t \mathbf{a}_i\\
      \mathbf{x}_{i+1} &= \mathbf{x}_i + \delta t \mathbf{v}_{i+1}
    \end{cases}
    $$

  \item As efficient as Explicit Euler
  \item Stable for small enough step sizes
  \item Conserves energy
    
  \end{itemize}
\end{frame}

\begin{frame}
  \frametitle{Other methods}
  \begin{itemize}
  \item Euler methods are first order integration methods
  \item Semi-implicit Euler method is good enough for video games
  \item There are more accurate methods~:
    Verlet (second order), Runge-Kutta 4 (fourth order)
  \item Too slow for games real-time requirements
    
  \end{itemize}
\end{frame}

\section{Constraints}
\begin{frame}
 \frametitle{What are constraints ?}
 \includegraphics[width=\textwidth]{watch.png}\\
  Joints, contacts, mechanical links...
\end{frame}

\begin{frame}
\frametitle {Real-life examples}
\begin{center}
\includegraphics[width = 0.2\textwidth]{humanjoint.png}
\includegraphics[width = 0.2\textwidth]{door.png}
\includegraphics[width = 0.2\textwidth]{elevator.png}
\includegraphics[width = 0.2\textwidth]{carcrash.png}
   \begin{exampleblock}{Examples}
    \begin{center}
     \begin{tabular}{c c c c}
       Human joint & Revolving door & Elevator & Collision \\
       Ball-socket & Hinge & Slider & Non-penetration
     \end{tabular}
     \end{center}
   \end{exampleblock}
   \includegraphics[width = 0.2\textwidth]{ballsock.png}
   \includegraphics[width = 0.2\textwidth]{hinge.png}
   \includegraphics[width = 0.2\textwidth]{slider.png}
   \includegraphics[width = 0.2\textwidth]{collision.png}
\end{center}
\end{frame}
\subsection{Modeling constraints}
\begin{frame}
 \frametitle{What are constraints ?}
 \begin{block}{Constraints}
  \begin{itemize}
    \item A constraint limits \textbf{degrees of freedom} of rigid bodies. \pause
    % NOTE : 2 RBs have 2 * 6 = 12 DOF.
    \item Engine computes \textbf{constraint forces} to apply to the bodies.
    \item Constraints must be \textbf{solved} by the physics engine to be satisfied.
  \end{itemize}
  \end{block}
\end{frame}


\begin{frame}
 \frametitle{Here comes the physics}
 \begin{itemize}
  \item A constraint can be written as an equation $C(\mathbf{q}) = 0$
  \item Example : Ball-socket $\mathbf{c_1} + \mathbf{r_1} - (\mathbf{c_2} + \mathbf{r_2}) = 0$\\
  \begin{center}\includegraphics[width = 0.4\textwidth]{jacobian1.png}\end{center}
  \pause
  \item We can derive and get a velocity equation $\dot{C} = 0$.
  \item $\mathbf{v_1} + (\mathbf{\omega_1} \times \mathbf{r_1}) - ( \mathbf{v_2} + (\mathbf{\omega_2} \times \mathbf{r_2})) = 0$
  \end{itemize}
\end{frame}

\begin{frame}
 \frametitle{Jacobians}
 \begin{block}{Jacobian : definition}
 \begin{itemize}
  \item Velocities constraints can be written as a scalar product
  \item $J\mathbf{v} = 0$
  \pause
  \item Virtual work theorem : $\mathbf{F_c} = J^T \mathbf{\lambda}$
  \end{itemize}
  \end{block}
\end{frame}

\begin{frame}
 \frametitle{Jacobians : the dirty details}
  \begin{center}\includegraphics[width = 0.2\textwidth]{pendulum.png}\end{center}
  \begin{itemize}
    \item $C(x,y) = x^2 + y^2 - L^2 = 0$
    \pause
    \item $\frac{\mathrm{d}C}{\mathrm{d}t} = \frac{ 1 }{\sqrt{x^2+y^2}} \begin{pmatrix} x & y\end{pmatrix} \begin{pmatrix} v_x \\ v_y\end{pmatrix}$
    \pause
    \item VWT : $\mathbf{F_c} = \lambda \begin{pmatrix}x\\y\end{pmatrix}$ 
  \end{itemize}
  How do we compute $\lambda$ ?
\end{frame}

\subsection{Physics solver}

\begin{frame}
 \frametitle{What is the solver doing ?}
 \begin{block}{Solver Algorithm}
 \begin{enumerate}
  \item Apply external impulses (e.g. gravity) $\bar{\mathbf{v}} = \mathbf{v} + M^{-1} \mathbf{P_{ext}}$
  \item Apply constraint impulses by solving for all $\lambda$
  \item Update positions from velocity.
 \end{enumerate}
 \end{block}
 Reminder :
 \begin{description}
    \item[Impulse] $\mathbf{P} = \mathbf{F} \delta t$
    \item[Newton] $\delta \mathbf{v} = M^{-1} \mathbf{P}$
 \end{description}
\end{frame}

\begin{frame}
 \frametitle{Finding $\lambda$}
  \begin{description}
   \item[Newton] $\mathbf{v'} = \bar{\mathbf{v}} + M^{-1} \mathbf{P_{c}}$
   \item[VWT] $\mathbf{P_c} = J^T \lambda$
   \item[Constraint] $J\mathbf{v'} = 0$
  \end{description}  
  \pause
  \begin{block}{Solution}
   \[ \lambda = -m_c (J \bar{\mathbf{v}})\]
   \[ m_c = \frac{1}{J M^{-1}J^T} \]
   $m_c$ is the virtual mass seen from the constraint
  \end{block}
\end{frame}

\begin{frame}[fragile]
 \frametitle{Solving all bodies}
  \begin{itemize}
   \item All the $\lambda_{i,j}$ equations = one big very sparse linear system.
   \pause
   \item We could solve it globally (slow)
   \pause
   \item Instead iterative methods are used.
  \end{itemize}
  \begin{lstlisting}
while (abs(vNext - vCurrent) > epsilon) {
 for (auto c : constraint) {
   solve(c, lambda);
  }
}
  \end{lstlisting}
\end{frame}


\begin{frame}
 \frametitle{Extensions}
 \begin{minipage}{0.78\linewidth}
  \begin{itemize}
   \item \textbf{Inequality constraint} ($C(\mathbf{q}) \geq 0$) for contacts.
   \item \textbf{Bias} in constraint ($J\mathbf{v} + b = 0$)
   \pause
   \item \textbf{Baumgarte stabilization} to avoid position drift.
  \end{itemize}
 \end{minipage}
% \vspace{}
 \begin{minipage}{0.2\linewidth}
   \includegraphics[width=0.9\textwidth]{blowup.png}
 \end{minipage}

\end{frame}





\section{Collision detection}

\begin{frame}
 \frametitle{Collision detection}
 \includegraphics[width=\textwidth]{apple.jpg}
\end{frame}


\begin{frame}
  \frametitle{Collision detection}
  
  \begin{block}{Why collision detection ?}
    \begin{itemize}
    \item Prevent character to fall through the floor
    \item Do not allow kills from behind a wall
    \item Check if a fired bullet touched a target
    \item Open a door when a character is walking toward it
    \item ...
    \end{itemize}
  \end{block}
  
\end{frame}
\subsection{Broadphase optimzation}

\begin{frame}[fragile]
  \frametitle{The bruteforce method}

  \begin{lstlisting}
    for_each object o1
      for_each object o2
        for_each triangle t1 in o1
          for_each triangle t2 in o2
            checkCollision( t1, t2 )
  \end{lstlisting}

  \pause
  \begin{itemize}
  \item ``Ok'' for two cubes
  \item Might be a bit slow when you have multiples characters averaging $150'000$ triangles each...
    \pause
  \item Eliminate what can not collide
  \item Test potential collisions more accurately
  \end{itemize}
  
\end{frame}

\begin{frame}
  \frametitle{Broad phase - Bounding Volume Hierarchies (BVH)}
  
  \begin{figure}
    \centering
    \includegraphics[width=0.8\linewidth]{bvh.png}
  \end{figure}

  \only<1> {
    \begin{itemize}
    \item Each object bounding volume is a leaf of a tree
    \item Those BV are grouped into small sets and new volumes are computed
    \item Same applies for the new volumes until the whole scene is enclosed within one volume
    \end{itemize}
  }

  \only<2> {
    \begin{itemize}
    \item Discriminate geometries against another by running through the tree
    \item Changing tree branch means no collision can occur
    \item Two leafs can then collide if there bounding volumes overlap
    \end{itemize}
  }
  
\end{frame}

\begin{frame}
  \frametitle{Broad phase - Sweep and Prune (SaP)}

  \begin{figure}
    \centering
    \includegraphics[width=0.5\linewidth]{sap.png}
  \end{figure}

  \begin{itemize}
  \item Project lower and upper bound of each object on $x$, $y$ and $z$ axis
  \item If bounds of two objects do not overlap for any axis, objects can not collide
  \item However if the overlap on all axes, objects can be tested
  \end{itemize}

\end{frame}

\subsection{Convex vs. convex with GJK}
\begin{frame}
  \frametitle{Collision between convex}
  \begin{figure}[p]
    \centering
    \includegraphics[width=0.8\linewidth]{convexcollision}
    \caption{Collision between convex shapes}
  \end{figure}
  \begin{itemize}
  \item Good enough approximation for collision detection
  \item Simple enough approximation for real-time requirements
  \end{itemize}
\end{frame}

\begin{frame}
  \frametitle{Minkowski difference}

  \begin{figure}[p]
    \centering
    \includegraphics[width=0.7\linewidth]{minkowski}
  \end{figure}
  \vspace{-1em}
  \visible<2-3> {
  $$
  Z = \left\{ y_j - x_i, x_i \in X, y_j \in Y \right\}
  $$
  }
  \vspace{-2em}
  \visible<3> {

  \begin{figure}[p]
    \centering
    \includegraphics[width=0.7\linewidth]{minkowskidetail}
  \end{figure}
  }

\end{frame}

\begin{frame}
  \frametitle{Point-convex distance}
  \begin{minipage}{0.6\linewidth}
    \begin{itemize}
    \item Minkowski difference properties~: convex-convex distance to point-convex distance
    \item Point-convex distance is solved by the GJK algorithm
    \item Trivia~: GJK stands for \emph{Gilbert-Johnson-Keerthi}
    \end{itemize}
  \end{minipage}
  \hfill
  \begin{minipage}{0.38\linewidth}
    \begin{figure}[p]
      \centering
      \includegraphics[width=0.9\linewidth]{gjk}
    \end{figure}
  \end{minipage}
\end{frame}

\subsection{Continuous Collision Detection}
\begin{frame}
  \frametitle{Continuous collision detection}

  \begin{figure}[p]
    \centering
    \only<1>{
      \includegraphics[width=0.9\linewidth]{ccd1}
    }
    \only<2>{
      \includegraphics[width=0.9\linewidth]{ccd2}
    }
  \end{figure}
\end{frame}

\begin{frame}
  \frametitle{Thicker wall ?}
  \begin{figure}[p]
    \centering
    \includegraphics[width=0.7\linewidth]{ccd3}
  \end{figure}
  \begin{itemize}
  \item Move physics engine issue to game design
  \item Greater velocity will still lead to tunelling
  \end{itemize}
  
\end{frame}

\begin{frame}
  \frametitle{Speculative contact points}
  
  \begin{minipage}{0.4\linewidth}
    \begin{figure}[p]
      \centering
      \includegraphics[width=0.9\linewidth]{ccd5}
    \end{figure}
  \end{minipage}
  \hfill
  \begin{minipage}{0.55\linewidth}
    \begin{itemize}
    \item Standard contact constraint~: $v_\text{rel} \cdot n \geq 0$
    \item Speculative contact constraint~: $v_\text{rel} \cdot n \geq -\frac{d}{\delta t}$
    \end{itemize}
  \end{minipage}

  \pause
  \begin{itemize}
  \item Leads to false positives
  \item More constraints
  \end{itemize}
  
\end{frame}

\begin{frame}
  \frametitle{Time of Impact}

  \begin{figure}[p]
    \centering
    \includegraphics[width=0.5\linewidth]{ccd6}
  \end{figure}

  \begin{itemize}
  \item Find the exact instant of the impact
  \item Perform integration / collision detections substeps
  \item More CPU time
  \end{itemize}
  
\end{frame}

\begin{frame}
  \frametitle{CCD, without}
  \begin{centering}
    \includemovie[autoplay=false]{\textwidth}{\textheight}{ccdoff.mp4}
  \end{centering}
\end{frame}

\begin{frame}
  \frametitle{CCD, with}
  \begin{centering}
    \includemovie[autoplay=false]{\textwidth}{\textheight}{ccdon.mp4}
  \end{centering}
\end{frame}

\section*{Conclusion}
\begin{frame}
\frametitle{Conclusion}
\includegraphics[width=\textwidth]{tank.png}
\end{frame}

\section*{Bibliography}

\begin{frame}
 \frametitle{Bibliography} 
  \begin{thebibliography}{9}
  \bibitem{Catto07} Erin \textsc{Catto} (Blizzard), \emph{Modeling and solving constraints}, Proceedings of GDC 2007.\href{http://box2d.org/files/GDC2007/GDC2007_Catto_Erin_Physics1.ppt}{[PPT]} 
  \bibitem{Catto07} Erin \textsc{Catto} (Blizzard), \emph{Computing Distance using GJK}, Proceedings of GDC 2010. \href{http://box2d.org/files/GDC2010/GDC2010_Catto_Erin_GJK.pdf}{[PDF]}
  \bibitem{Ericson04} Christer \textsc{Ericson} (Sony), \emph{Real-time collision detection}. CRC Press, 2004.
  \bibitem{Shabana09} Ahmed A. \textsc{Shabana} (U.Illinois), \emph{Computational dynamics}. John Wiley \& Sons, 2009
  \end{thebibliography}
\end{frame}


\end{document}
